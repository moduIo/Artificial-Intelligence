\documentclass[11pt, oneside]{article}   	% use "amsart" instead of "article" for AMSLaTeX format
\usepackage{geometry}                		% See geometry.pdf to learn the layout options. There are lots.
\geometry{letterpaper}                   		% ... or a4paper or a5paper or ... 
%\geometry{landscape}                		% Activate for rotated page geometry
%\usepackage[parfill]{parskip}    		% Activate to begin paragraphs with an empty line rather than an indent
\usepackage{graphicx}				% Use pdf, png, jpg, or eps§ with pdflatex; use eps in DVI mode
								% TeX will automatically convert eps --> pdf in pdflatex
\usepackage{amssymb}
\usepackage{amsmath}
\usepackage{algorithm}% http://ctan.org/pkg/algorithm
\usepackage{algpseudocode}% http://ctan.org/pkg/algorithmicx
\usepackage{mathrsfs}

%SetFonts

%SetFonts


\title{CSE537 HW1}
\author{Tim Zhang}
%\date{}							% Activate to display a given date or no date

\begin{document}
\maketitle
%\section{}
%\subsection{}

\section{Heuristics}
My implementation of A* used the general tree search algorithm as a base and as such I will only show the admissibility of my chosen heuristics.

\subsection{Misplaced Tiles Heuristic}
The misplaced tiles heuristic is derived by relaxing the formal problem definition of the $n \times n$ puzzle to allow tiles to be moved to any space regardless of if the space is empty or the relative location of the space with respect to the current location of the tile.

The misplaced tiles heuristic is admissible since in the best case a tile is misplaced by the correct single empty adjacent square or it is already in the correct position.  When this occurs the actual distance the tile needs to move is equal to the estimated distance provided.  That is, it is either 1 or 0 respectively.  

For all other possibilities the number of moves required to correctly position the misplaced tile will be greater than the single move estimated due to the fact that the blank will have to move at least twice to be correctly adjacent to the misplaced tile.  Since this is the case the heuristic will always underestimate the number of required moves to solve the puzzle and is therefore admissible.

\subsection{Manhattan Distance Heuristic}
The Manhattan distance is also derived by relaxing the formal definition of the problem.  In this relaxation a tile may move to an adjacent position regardless of if it is occupied or not.

This heuristic is admissible since by not considering the occupancy of an adjacent tile we disregard a series of moves needed in the real problem for each Manhattan move.  In the best case the adjacent tile is both empty and exactly the correct position and there is only a single misplaced tile, when this occurs the Manhattan distance is the same as the true distance.  Otherwise the true distance will always be longer since it will take some series of moves to achieve the same effect.

\pagebreak{}

\section{Memory Issues}

Even with the tree search implementation of A* the memory requirements become prohibitive in the $4 \times 4$ case.  This is due to the fact that the frontier is kept in memory and nodes are not goal tested until they are considered for expansion.

The amount of memory needed to solve a solvable instance of the 15-puzzle is bounded by $O(4^d)$ where $d$ is the depth of the optimal solution since each state has at most 4 possible successors.  By ignoring reversible actions we can reduce this to $O(3^d)$ nodes which we need to keep in memory.  So for each best node we consider we add 3 more nodes to the frontier in the worst case.

Empirically, a single instance of the $4 \times 4$ puzzle used up over 11GB of memory on my computer and was terminated on my Windows machine where processes have a memory limit.

\pagebreak{}

\section{Iterative Deepening A*}

IDA* is a recursive search algorithm which is essentially an uninformed DFS which kills branches when the current node has an $f$-cost greater than the current depth limit.  When this happens the parent of the node will keep track of the $f$-cost of the pruned node and continue the recursion through each of it's successors.  Finally, when all successors are accounted for, if there is not a solution found the node will return the smallest $f$-cost which exceeded the depth limit.  If a solution is found it is returned, and if no solution is found and there was no node which exceeded the limit then there is no solution on the branch.  The original depth limit is initialized to the $f$-cost of the start state.

IDA* is both optimal and complete given that the heuristic function is admissible.  This is because IDA* will search the entire space up to and including the contour where the $f$-values are equal to the cost of the optimal solution.

The time complexity of IDA* is $O(b^{\epsilon d})$ where $d$ is the depth of the optimal solution, $\epsilon$ is the relative error and $b$ is the branching factor this is because the time complexity of IDA* is bounded by the iteration which finds the optimal solution.  In this iteration, in the worst case, every node will be considered and there are $b^{\epsilon d}$ nodes in total.

The space complexity of IDA* is $O(d)$ where $d$ is the depth of the optimal solution.  This is because there will be at most $d$ nodes on a path and (recursive) IDA* will only keep a single path in memory at a time.

\pagebreak{}

\section{A* Performance}
See section 4.3 for initial state references.

\subsection{Manhattan Distance Heuristic}
$\newline$
\begin{center}
 \begin{tabular}{||c c c c||} 
 \hline
 Initial State & States Explored & Time (ms) & Depth \\ [0.5ex] 
 \hline\hline
 1 & 986 & 48.69 & 20 \\ 
 \hline
 2 & 1612 & 84.134 & 24 \\ 
  \hline
 3 & 29271 & 1749.189 & 24 \\ 
  \hline
 4 & 3326 & 176.70600000000002 & 24 \\ 
  \hline
 5 & 219 & 12.143 & 20 \\ 
  \hline
 6 & 2900 & 161.10299999999998 & 24 \\ 
  \hline
 7 & 536 & 27.78300000000001 & 20 \\ 
  \hline
 8 & 1064 & 56.297 & 20 \\ 
  \hline
 9 & 2294 & 124.398 & 22 \\ 
  \hline
 10 & 8362 & 491.947 & 26 \\ 
  \hline
11 & 39 & 1.948999999999998 & 14 \\ 
  \hline
 12 & 2115 & 125.49600000000001 & 24 \\ 
  \hline
 13 & 1133 & 60.93800000000001 & 22 \\ 
  \hline
 14 & 946 & 47.239000000000004 & 20 \\ 
  \hline
 15 & 30 & 2.2449999999999974 & 14 \\ 
  \hline
 16 & 705 & 36.932 & 20 \\ 
  \hline
 17 & 487 & 24.528999999999996 & 22 \\ 
  \hline
 18 & 2206 & 123.22400000000002 & 24 \\ 
  \hline
 19 & 1018 & 56.41400000000001 & 22 \\ 
  \hline
 20 & 546 & 29.629999999999995 & 20 \\ [1ex] 
 \hline
\end{tabular}
\end{center}
\pagebreak{}

\subsection{Misplaced Tiles Heuristic}
$\newline$
\begin{center}
 \begin{tabular}{||c c c c||} 
 \hline
 Initial State & States Explored & Time (ms) & Depth \\ [0.5ex] 
 \hline\hline
 1 & 7493 & 317.771 & 20 \\
 \hline
 2 & 52059 & 2473.1639999999998 & 24 \\ 
  \hline
 3 & 1018590 & 64340.001000000004 & 30 \\ 
  \hline
 4 & 43662 & 2359.032 & 24 \\ 
  \hline
 5 & 4369 & 198.59499999999997 & 20 \\ 
  \hline
 6 & 41415 & 2226.945 & 24 \\ 
  \hline
 7 & 5062 & 234.983 & 20 \\ 
  \hline
 8 & 6865 & 338.041 & 20 \\ 
  \hline
 9 & 17671 & 909.362 & 22 \\ 
  \hline
 10 & 125701 & 7530.735 & 26 \\ 
  \hline
11 & 224 & 9.323 & 14 \\ 
  \hline
 12 & 40884 & 2108.9350000000004 & 24 \\ 
  \hline
 13 & 18447 & 933.249 & 22 \\ 
  \hline
 14 & 5720 & 266.027 & 20 \\ 
  \hline
 15 & 225 & 10.811 & 14 \\ 
  \hline
 16 & 5187 & 238.56500000000003 & 20 \\ 
  \hline
 17 & 13525 & 660.457 & 22 \\ 
  \hline
 18 & 42101 & 2208.1290000000004 & 24 \\ 
  \hline
 19 & 18372 & 941.467 & 22 \\ 
  \hline
 20 & 5513 & 255.20099999999996 & 20 \\ [1ex] 
 \hline
\end{tabular}
\end{center}
\pagebreak{}

\subsection{Initial States}
Shown as an list with 0 being the blank tile.\\
1. (5 1 6 3 0 8 7 4 2)\\
2. (8 5 6 7 0 4 3 1 2)\\
3. (0 5 4 8 6 7 2 3 1)\\
4. (8 1 2 7 5 4 3 6 0)\\
5. (4 7 0 8 6 1 5 3 2)\\
6. (5 6 1 8 4 2 0 3 7)\\
7. (8 1 0 4 7 3 5 6 2)\\
8. (2 7 1 4 0 5 8 3 6)\\
9. (6 2 3 5 1 8 7 4 0)\\
10. (2 7 1 6 5 8 0 4 3)\\
11. (0 8 5 1 6 2 4 7 3)\\
12. (5 4 0 7 8 1 3 2 6)\\
13. (6 7 2 3 0 8 4 1 5)\\
14. (0 2 6 5 1 3 8 4 7)\\
15. (6 2 0 1 4 8 7 3 5)\\
16. (5 2 6 1 7 4 0 8 3)\\
17. (0 2 7 8 6 3 5 1 4)\\
18. (5 3 7 8 4 6 1 2 0)\\
19. (6 1 8 5 0 3 2 4 7)\\
20. (4 8 5 2 3 6 7 1 0)
\end{document}  